\documentclass[12pt]{article}

\usepackage[margin=1in]{geometry}
\usepackage{lipsum}
\usepackage{abstract}
\usepackage{amsfonts}
\usepackage{amssymb}
\usepackage{amsthm}
\usepackage{amsmath}
\usepackage{bm}
\usepackage{bbm}
\usepackage[square]{natbib}
\usepackage{hyperref}
\hypersetup{
    colorlinks=true,
    linkcolor=blue,
    citecolor=blue,
    anchorcolor=blue,
    filecolor=magenta,
    urlcolor=cyan
}

\begin{document}

\section{Introduction}

Let \(y_{ijsge}\) denote the level of bilateral migration of individuals of
gender \(g\) and education level \(e\) from country \(i\) to sector \(s\) of
country \(j\). Let \(x_{ij}\) denote the level of foreign aid flowing from
country \(i\) to country \(j\).  In it's most general form, our model seeks to
understand the relationship.

\begin{equation}
    y_{ij} = f(x_{ij}, Z_{i}, Z_{j}, \delta_{s}, \gamma_{g}, \lambda_{e})
\end{equation}

The model links two bilateral flows (i.e., foreign aid and migration) and draws
inspiration form ``gravity'' models of international trade.  A linear
specification for pooled ordinary least squares with fixed effects is

\begin{equation}
    y_{ij} = \alpha + \beta x_{ij} + Z_{i} + Z_{j} + \delta_{s} + \gamma_{g} + \lambda_{e}
\end{equation}

where \(Z_{i}\) are donor fixed effects, \(Z_{j}\) are receiver fixed effects,
\(\delta_{s}\) are sector fixed effects, \(\gamma_{s}\) are sector fixed effects

\section{Assumptions and Caveats}

\begin{enumerate}
    \item The relationship between foreign aid and migration is linear.
    \item Migration flows observed round year 2000 are proportional to flows
        observed in 2011.  This assumption is a result of the nature of the
        observed data: migration is observed around year 2000 at the
        continent-country level and foreign aid is observed annually from 2011
        to 2020 at the country-country level.
    \item The bilateral migration data is observed only at the continent to
        country level.  Therefore we need to assume that the flow of foreign aid
        from country \(i\) to country \(j\) in continent \(c\) can be directly
        linked to the flow of migrants from continent \(c\) to country \(i\).
    \item We make no assumptions on the direction of causality in this
        specification.
\end{enumerate}

\section{Results}

Overall, the estimates from our specifications consistently suggest \textbf{a
negative relationship between foreign aid and bilateral migration flows}.  All
else being held equal, an increase in the level of foreign aid from country
\(i\) to country \(j\) is associated with a decrease in the level of bilateral
migration from country \(i\) to country \(j\).  In our preferred specification,
increasing foreign aid from country \(i\) to country \(j\) by \$1 million (2011
USD) is associated with 134 fewer migrants from country \(j\) to country \(i\),
on average.  This result is robust to the inclusion of all of our fixed effects.

\end{document}
